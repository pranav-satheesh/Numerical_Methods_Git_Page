\documentclass[10pt,a4paper]{article}
\usepackage[utf8]{inputenc}
\usepackage{amsmath}
\usepackage{amsfonts}
\usepackage{amssymb}
\usepackage{graphicx}
\title{ODE: Advanced Methods}
\begin{document}
\maketitle

\section*{RUNGE-KUTTA METHODS}

\subsection*{Second Order Runge-Kutta}

Our general ODE takes the form,

\begin{equation}\label{eq1}
\frac{d\textbf{x}}{dt}=\textbf{f}(\textbf{x}(t),t)
\end{equation}

where the vector \textbf{x}(t) is the desired solution.

Consider the kepler problem of an orbit in the x-y plane.We have,

$$\textbf{x}(t)=[r_x(t),r_y(t),v_x(t),v_y(t)]$$
and

\begin{align*}
\textbf{f}(\textbf{x}(t),t)&=\left[ \frac{d r_{x}}{dt},\frac{d r_{y}}{dt},\frac{d v_{x}}{dt},\frac{d v_y}{dt}\right]\\
&=\left[v_x (t), v_y (t), F_x(t)/m,F_y(t)/m \right]
\end{align*}


The first Runge-Kutta formula is,
\begin{equation}
x(t+\tau) = x(t) + \tau f\left(x^*\left(t+\frac{\tau}{2}\right),t+\frac{\tau}{2}\right)
\end{equation}
where
$$ x^*(t+\tau/2) = x(t) + \frac{\tau}{2}f(x(t),t)$$

To explain why we are doing this, we need to step back and look at the taylor expansion in one-variable case.

\begin{align*}
x(t+\tau)&=x(t) + \tau\frac{d x(\zeta)}{dt}\\
&=x(t) + \tau f(x(\zeta),\zeta)
\end{align*}

This expansion is exact for some value $\zeta$ between t and $t+\tau$.\\
\begin{enumerate}
\item The Euler formula takes $\zeta=t$.
\item Euler-Cormer uses $\zeta=t$ in the velocity and$\zeta=t+\tau$ in the position euation.
\item RungeKutta tries to use $ \zeta=t+\tau/2 $, since that is probably a better guess.However since $x(t+\frac{\tau}{2})$ is not known, we use an Euler step to compute $x*(t+\tau/2)$
\end{enumerate}


\subsection*{Fourth-Order Runge-Kutta}

In practice, the most commonly used method is the following fourth-order Runge-Kuta fomula:
\begin{equation}
x(t+\tau) = x(t)+\frac{1}{6}\tau[F_1 + 2 F_2 + 2 F_3 + F_4]
\end{equation}

where

\begin{align*}
F_1&=f(x,t)\\
F_2&=f\left(x+\frac{1}{2}\tau F_1,t+\frac{1}{2}\tau \right)\\
F_3&=\left(x+\frac{1}{2}\tau F_2,t+\frac{1}{2}\tau \right)\\
F_4&= f(x+\tau F_3,t+\tau)
\end{align*}

\subsection*{Adaptive Runge-Kutta Method}

In certain problems, we have to vary $\tau$. We have a rough idea of what $\tau$ should be; now we have to selecta $\tau_{min}$ and $\tau_{max}$ and find a way to switch betwen them. If we do this manually, it would be worse than just doing the brute-force calculation with a small time step.

Adaptive programs monitor the time step such ythat user given accuracy is always mainatained. An example of such a methos is as follows. Suppose, you are going from the state $x(t)$ to $x(t+\tau)$, we can do it in two ways.

\begin{itemize}
\item Take a big step from $x(t)$ to $x(t+\tau)$. Call this $x_b(t+\tau)$.
\item Now, repeat the same calculation but with a time step of $\tau/2$. Call this value you got from two jumps of $\tau/2$, $x_s(t+\tau)$
\end{itemize}

Now, find the difference between these two calculation. that will estimate the local truncation error. If the error is tolerable, we ggo with the timestep $\tau$. otherwiise, we choose a lower time step and repeat the clauclation till the optimal time step is found. The estimated local truncation error can guife us selectung a new time step for the next iteration.

Call $\Delta$ the local truncation errror. we know that for RK4 method, $\Delta \propto \tau^5$.
Suppose the current time step $\tau_{old}$ gave an error of 
$\Delta_{c}=|x_b - x_s|$. Given that we want the error to be less than or equal to the user-specified ideal error, $\Delta_i$, then, the estimate for the new time step is:

\begin{equation}
\tau_{est} = \tau \|\frac{\Delta_i}{Delta_c}|^{1/5}
\end{equation}

Since this is only an estimate, the new time step is $\tau_{new} = S_1 \tau_{est}$, where $S_1 < 1$. We also need a safety factor $S_2 >1$ just to be sure that the program won't astly raise or lower the time step

\begin{equation}
  \tau_{new} =
    \begin{cases}
      S_2 \tau_{old} & \text{if S_1 \tau_{est}> S_2 \tau_{old}\\
      \tau_{old}/S_2 & \text{if S_1 \tau_{est}<\tau_{old}/S_2}\\
      S_1 \tau_{est} & \text{otherwise}
    \end{cases}       
\end{equation}\\



\subsection*{CHAOS in Lorenz Model}

(*still in development*)

\begin{align}
\frac{dx}{dt}&=\sigma(y-x)\\
\frac{dy}{dt}&= rx-y-xz\\
\frac{dz}{dt}&= xy-bz
\end{align}

T

\end{document}