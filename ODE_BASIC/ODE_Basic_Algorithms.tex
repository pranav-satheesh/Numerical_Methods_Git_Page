\documentclass[10pt,a4paper]{article}
\usepackage[utf8]{inputenc}
\usepackage{amsmath}
\usepackage{amsfonts}
\usepackage{amssymb}
\usepackage{graphicx}
\title{ODE-Basic Algorithms}
\begin{document}
\maketitle


\section{Euler Method}

\subsection{Forward Derivative}


\begin{enumerate}
\item The formal definition of the derivative is,

$$ f^{'}(t) = lim_{\tau\rightarrow0} \frac{f(t+\tau)-f(t)}{\tau}$$

\item From the definition of Taylor's theorem, we can write:

$$ f^{'}(t) = \frac{f(t+\tau)-f(t)}{\tau} -\frac{1}{2} \tau f^{''}(\zeta)$$

where $t \leq \zeta \leq t+\tau$. This is the \emph{right derivative} or \emph{forward derivative formula}. The last term is the truncation error which is of the order of $\tau$ here.
\end{enumerate}


\subsection{Euler's Method}

Consider the equations of motion here, which I want to solve numerically,

$$ \frac{d \textbf{v}}{dt} = \textbf{a}(\textbf{r},\textbf{v})$$

$$\frac{d\textit{r}}{dt} = \textbf{v}$$

Using the forward derivative equation, we can write these as,

\begin{align*}
\textbf{v}(t+\tau)&= \textbf{v}(t) + \tau\textbf{a}(\textbf{r}(t),\textbf{v}(t))+\mathcal{O}(\tau^{2})\\
\textbf{r}(t+\tau) &= \textbf{r}(t) + \tau\textbf{v}(t) + \mathcal{O}(\tau^{2})
\end{align*}



Our notation will be,
$f_n = f(t_n)$,
$t_n = (n-1)\tau $

The Euler method equations become,

\begin{align*}
\textbf{v}_{n+1}&=\textbf{v}_{n}+\tau \textbf{a}_{n}\\
\textbf{r}_{n+1}&= \textbf{r}_{n} + \tau \textbf{v}_{n}
\end{align*}


\subsection{Euler-Cromer Method}

Instead of $ v_{n} $ in the quation, we put the modified $v_{n+1}$

\begin{align*}
\textbf{v}_{n+1}&=\textbf{v}_{n}+\tau \textbf{a}_{n}\\
\textbf{r}_{n+1}&= \textbf{r}_{n} + \tau \textbf{v}_{n+1}
\end{align*}

The truncation is still of $\mathcal{O(\tau^{2})}$.

\subsection{Midpoint Method}

We can have the midpoint of velocities between vn and vn+1

\begin{align*}
\textbf{v}_{n+1}&=\textbf{v}_{n}+\tau \textbf{a}_{n}\\
\textbf{r}_{n+1}&= \textbf{r}_{n} + \tau \frac{ \textbf{v}_{n+1} +\textbf{v}_{n}}{2}
\end{align*}

Plugging the velcoity equation into the position equation, we see that

$$ \textbf{r}_{n+1} = \textbf{r}_{n} + \tau \textbf{v}_{n} + \tau \textbf{v}_{n} + \frac{1}{2} \textbf{a}_{n} \tau^{2}$$

The order is $\mathcal{O}(\tau^{3})$



global error = $ N_{\tau}$ x (local error) = (T/$ \tau $)$ \mathcal{O}(\tau^{n}) = T\mathcal{O}(\tau^{n-1})$

















\end{document}